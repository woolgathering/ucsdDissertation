\chapter{Extraction of Periodic Bases}\label{chap:extraction}

In contrast to other literature on the periodicity transform, where the primary and concern is the detection of periodicities and extraction of a single periodic waveform, we are concerned here with also hearing the basis vectors themselves.

We want to get the periodic bases out so that they are not necessarily exactly periodic, as is the concern when searching for the periods themselves. Suppose there is a signal, $x$, that is composed of of two signals, $x_0$ an $x_1$ that contain components at each of their factors down to 1 (i.e. they are not exactly periodic) and whose fundamental periods are not mututally prime. Let $C = \text{factors}(x_0) \cap \text{factors}(x_1)$ where $\text{factors}()$ is Equation \eqref{eq:getFactors}. We can see that the two signals share components at periods $c \in C$ and that in order to extract the periods, one must determine how much of each common component is contained in each signal.


% We need to create redistribution matricies in order to properly distribute the various factors of Q into all other possibilities. That is, suppose two periodic functions $f$ and $g$ of periods $p_f$ and $p_g$ have a common factor $d$. We can detect the $d$-component of both $f$ and $g$ independently but when it comes time to reconstruct, how do we know how much of the $d$-component goes into f and how much of the $d$-component goes into $g$? We can again solve this by formulating a convex program and solving in the standard format or by least-squares. In this case, however, we will concatenate the period coefficients, zero-padded if necesssary, as the input to the convex program.

In order to solve this, we also set up a convex program whose input vector is the concatenation of the periods $q \in Q$ using their respecitve values in $y$ (output of Equation \eqref{eq:detection:convex}), zero padding as necessary so that each period is ``filled out''. Let this concatenation of periods be $w$.

\section{Construction of $w$}
An important note here is the order in which the periodic vectors were placed in $\bm{M}_Q$ since this determintes their corresponding coefficeints in the output of the convex program, $y$. Thus, the method by which $\bm{M}_Q$ was constructed is of importance and determines the method by which we construct $w$.

    \subsection{Separate Common Factors}
    The method here corresponds to the construction method in \ref{detection:A:separateFactors}. We know from Equation \eqref{eq:A:commonFactors:cp} that each component in $P$ as defined in Equation \eqref{eq:A:commonFactors:P} contains exactly $c_{p}$ columns which correspond to exactly $c_{p}$ coefficients in the output vector $y$. The position of these coefficients in $y$ corresponds to the order in which they were added to create $\bm{M}_Q}$.
    We then know that for $p_0$ the coefficeints that correspond to this period in $y$ are the first $c_{p_0}$ values. Likewise, for $p_1$, the coefficeints in $y$ are the next $c_{p_1}$ values. Generalizing, we see that the values in $y$ that correspond to the $i$th period from $P$ are:
    \begin{align}
        w_{p_i}^{*} &= y_j \text{, } \alpha_{i-1} \leq j \leq \alpha_{i} \\
        \alpha_{i} &= \sum_{k = 0}^{i} c_{p_k}
    \end{align}
    Notice that $w_{p_i}^{*}$ is not guaranteed to be of legnth $p_i$.\footnote{In fact, in the case of separating out factors to form $\bm{M}_Q$, it will never be.} We therefore zero-pad $w_{p_i}^{*}$ to get $w_{p_i}$:
    \begin{align}
        w_{p_i} = \text{zp}(w_{p_i}^{*}, p_i)
    \end{align}
    where $\text{zp}(x, N)$ is a function which concatenates zeros to the end of $x$ until $|x| = N$. $w$ is then formed by concatenation:
    \begin{align}
        w = \begin{bmatrix}
            w_{p_0} & w_{p_1} & w_{p_2} & \hdots & w_{p_n}
        \end{bmatrix}
    \end{align}

    \subsection{Progressive Concatenation}
    The method here corresponds to the construction method in \ref{detection:A:progressiveConcatenation}.


\section{Creating redistribution matricies}
Once the periods have been concatenated, we then have to construct $\bm{A}$ for input into the convex program. In contrast to $\bm{A}$ in Chapter \ref{chapter:detection}, 







% Define $P$:
% \begin{align*}
%     P &= \Bigg( \bigcup_{\forall p, q \in Q} f(p) \cap f(q) \Bigg) \cup Q
% \end{align*}
% i.e. the union of the intersections of the factors of all the elements in $Q$.
%
% To find the column dimensionality, $c_{p}$, of the basis matrix $\bm{B_p}$ for $p \in P$:
% \begin{align*}
%     % c_{p} &= p - \sum_{f : f(p) \cap P \setminus p} \varphi(f)
%     c_{p} &= p - \sum_{d|p \in P \setminus p} \varphi(d)
% \end{align*}
%
% %% If N is the columns
% % We then define $\bm{A}$ as:
% % \begin{align*}
% %     \bm{A} &= \begin{bmatrix}
% %             \bm{B}_{p_0} \\
% %             \bm{B}_{p_1} \\
% %             \vdots \\
% %             \bm{B}_{p_n} \\
% %         \end{bmatrix}
% %         , \text{ } \forall p \in P
% % \end{align*}
% % where there are $n$ elements in $\bm{P}$.
%
% To find the total column dimensionality, $c_{\boldsymbol{A}}$ of the matrix $\bm{A}$:
% \begin{align*}
%     c_{\bm{A}} &= \sum_{p \in P} c_p
% \end{align*}

% In this way, we can be assured that not only are the basis vectors in $\bm{A}$ linearly independent, but also that all traces of some period $q$ are removed from the residual signal $h$.

% \subsection{This also works}
% We can also get a basis matrix for the periods in $q$ by using the same equation as above that uses $varphi(n)$ by creating $\bm{A}$ along the way and keeping track of which periods (and factors of the periods) we've added along the way.
%
% Let $F = \emptyset$. To find the column dimensionality, $c_q$, for $q \in Q$:
% \begin{align}
%     c_q &= q - \sum_{d|q \in F} \varphi(d)
% \end{align}
%
% We then add elements to $F$ every time we add a basis matrix $\bm{B}_q$ to $\bm{A}$:
% \begin{align*}
%     F &= \{ F \cup f(q) \}_{\neq}
% \end{align*}
% i.e. every time we process a period $q$, add all the factors of $q$ to $F$, removing duplicates.
%
% Repeat this process for every $q \in Q$, adding to $F$ along the way. (This means that we need to keep track of which factors we've added along the way which makes it uglier in some ways, better in others.)
%
% Therefore, we then define $\bm{A}$ as:
% \begin{align*}
%     \bm{A} &= \begin{bmatrix}
%             \bm{B}_{q_0} &
%             \bm{B}_{q_1} &
%             \hdots &
%             \bm{B}_{q_n}
%         \end{bmatrix}
%         , \text{ } \forall q \in \bm{Q}
% \end{align*}
% where there are $n$ elements in $\bm{Q}$.
%
% \section{Not a real section but describing getting the real periods back}
