\chapter{Decomposing and Resynthesizing Audio}
An elementary use of the PT in audio is for the decomposition and resynthesis of audio signals.

A requirement of a PT on audio data is its ability to handle non-integer periods, in contrast to something like DNA sequencing where periods are guarenteed to be integers.

\section{Synthetic Audio}

    \subsection{Sine Waves}

    \subsection{Sine Waves + Noise}

    \subsection{Mixed Waves}
    An interesting phenomenon that is not necessarily unexpected is ``waveform interference'' whereby discovered waveforms bear features of other waveforms in the signal. Below is a signal comprised of a sine ($p = 100$), a square ($p = 171$), and a sawtooth ($p = 203$) which was analyzed using the QO method:

    Figure XXX shows the orignal periods compared against the periodic components derived from the QO. Note that despite their mututal prime-ness, the waveforms ``interfere'' with one another with regard to their shapes.




\section{Recorded Audio}


\section{PEAQ Comparisons}
