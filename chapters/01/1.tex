\chapter{Quadratically Optimized Periodic Decomposition} \label{chap:qo}

The fundamental difference between this and \cite{sethares1999periodicity} is the difference in computing the residual. That is, one can frame the residual computation as a quadratic program which allows one to more accurately and find the periodic components.

\section{Find the matrix $\bm{A}$}
Assume we have the following:
\begin{align*}
x &= \text{input sequence} \\
Q &= \text{set of periods posited to be in } x \\
f(n) &= \{f \in N : f|n \} \text{ (i.e. a function that the set of all factors of $n$)} \\
\varphi(n) &= \text{Euler's totient function}
\end{align*}

Let $\bm{A}$ be a ``basis matrix'', or a matrix of linearly independent columns which span the space of the sequence $x$ with periods $Q$. The column dimensionality, $c_{\bm{A}}$, of $\bm{A}$ is:
\begin{align*}
    c_{\bm{A}} = \sum_{R_i \in R} \varphi(R_i)
\end{align*}
where $R$ is the union of all the factors of the periods in $Q$:
\begin{align*}
    R = \bigcup_{q_i \in Q} f(q_i)
\end{align*}

In creating the basis matrix $\bm{A}$, one must ensure that each periodic component is counted \emph{once}. That is, each periodic component $q$ also contains the components $d|q$ and one must construct the basis matrix with this in mind, taking care to remove the component from a subspace if it is already present. The method for doing this is as follows:




Define $P$:
\begin{align*}
    P &= \Bigg( \bigcup_{\forall p, q \in Q} f(p) \cap f(q) \Bigg) \cup Q
\end{align*}
i.e. the union of the intersections of the factors of all the elements in $Q$.

To find the column dimensionality, $c_{p}$, of the basis matrix $\bm{B_p}$ for $p \in P$:
\begin{align*}
    % c_{p} &= p - \sum_{f : f(p) \cap P \setminus p} \varphi(f)
    c_{p} &= p - \sum_{d|p \in P \setminus p} \varphi(d)
\end{align*}

%% If N is the columns
% We then define $\bm{A}$ as:
% \begin{align*}
%     \bm{A} &= \begin{bmatrix}
%             \bm{B}_{p_0} \\
%             \bm{B}_{p_1} \\
%             \vdots \\
%             \bm{B}_{p_n} \\
%         \end{bmatrix}
%         , \text{ } \forall p \in P
% \end{align*}
% where there are $n$ elements in $\bm{P}$.

We then define $\bm{A}$ as:
\begin{align*}
    \bm{A} &= \begin{bmatrix}
            \bm{B}_{p_0} &
            \bm{B}_{p_1} &
            \hdots &
            \bm{B}_{p_n}
        \end{bmatrix}
        , \text{ } \forall p \in P
\end{align*}
where there are $n$ elements in $\bm{P}$.

To find the total column dimensionality, $c_{\boldsymbol{A}}$ of the matrix $\bm{A}$:
\begin{align*}
    c_{\bm{A}} &= \sum_{p \in P} c_p
\end{align*}

% In this way, we can be assured that not only are the basis vectors in $\bm{A}$ linearly independent, but also that all traces of some period $q$ are removed from the residual signal $h$.

\subsection{This also works}
We can also get a basis matrix for the periods in $q$ by using the same equation as above that uses $varphi(n)$ by creating $\bm{A}$ along the way and keeping track of which periods (and factors of the periods) we've added along the way.

Let $F = \emptyset$. To find the column dimensionality, $c_q$, for $q \in Q$:
\begin{align}
    c_q &= q - \sum_{d|q \in F} \varphi(d)
\end{align}

We then add elements to $F$ every time we add a basis matrix $\bm{B}_q$ to $\bm{A}$:
\begin{align*}
    F &= \{ F \cup f(q) \}_{\neq}
\end{align*}
i.e. every time we process a period $q$, add all the factors of $q$ to $F$, removing duplicates.

Repeat this process for every $q \in Q$, adding to $F$ along the way. (This means that we need to keep track of which factors we've added along the way which makes it uglier in some ways, better in others.)

Therefore, we then define $\bm{A}$ as:
\begin{align*}
    \bm{A} &= \begin{bmatrix}
            \bm{B}_{q_0} &
            \bm{B}_{q_1} &
            \hdots &
            \bm{B}_{q_n}
        \end{bmatrix}
        , \text{ } \forall q \in \bm{Q}
\end{align*}
where there are $n$ elements in $\bm{Q}$.

\section{Not a real section but describing getting the real periods back}
