%
%
% UCSD Doctoral Dissertation Template
% -----------------------------------
% https://github.com/ucsd-thesis/ucsd-thesis
%
%
% ----------------------------------------------------------------------
% WARNING:
%
%   This template has not endorced by OGS or any other official entity.
%   The official formatting guide can be obtained from OGS.
%   It can be found on the web here:
%   http://grad.ucsd.edu/_files/academic-affairs/Dissertations_Theses_Formatting_Manual.pdf
%
%   No guaranty is made that this LaTeX class conforms to the official UCSD guidelines.
%   Make sure that you check the final document against the Formatting Manual.
%
%   That being said, this class has been routinely used for successful
%   publication of doctoral theses.
%
%   The ucsd.cls class files are only valid for doctoral dissertations.
%
%
% ----------------------------------------------------------------------
% GETTING STARTED:
%
%   Lots of information can be found on the project wiki:
%   http://code.google.com/p/ucsd-thesis/wiki/GettingStarted
%
%
%   To make a pdf from this template use the command:
%     pdflatex template
%
%
%   To get started please read the comments in this template file
%   and make changes as appropriate.
%
%   If you successfully submit a thesis with this package please let us
%   know.
%
%
% ----------------------------------------------------------------------
% KNOWN ISSUES:
%
%   Currently only the 12pt size conforms to the UCSD requirements.
%   The 10pt and 11pt options make the footnote fonts too small.
%
%
% ----------------------------------------------------------------------
% HELP/CONTACT:
%
%   If you need help try the ucsd-thesis google group:
%   http://groups.google.com/group/ucsd-thesis
%
%
% ----------------------------------------------------------------------
% BUGS:
%
%   Please report all bugs at:
%   https://github.com/ucsd-thesis/ucsd-thesis/issues
%
%
% ----------------------------------------------------------------------
% More control of the formatting of your thesis can be achieved through
% modifications of the included LaTeX class files:
%
%   * ucsd.cls    -- Class file
%   * uct10.clo   -- Configuration files for font sizes 10pt, 11pt, 12pt
%     uct11.clo
%     uct12.clo
%
% ----------------------------------------------------------------------



% Setup the documentclass
% default options: 12pt, oneside, final
%
% fonts: 10pt, 11pt, 12pt -- are valid for UCSD dissertations.
% sides: oneside, twoside -- note that two-sided theses are not accepted
%                            by OGS.
% mode: draft, final      -- draft mode switches to single spacing,
%                            removes hyperlinks, and places a black box
%                            at every overfull hbox (check these before
%                            submission).
% chapterheads            -- Include this if you want your chapters to read:
%                              Chapter 1
%                              Title of Chapter
%
%                            instead of
%                              1 Title of Chapter
\documentclass[12pt,chapterheads]{ucsd}



% Include all packages you need here.
% Some standard options are suggested below.
%
% See the project wiki for information on how to use
% these packages. Other useful packages are also listed there.
%
%   http://code.google.com/p/ucsd-thesis/wiki/GettingStarted



%% AMS PACKAGES - Chances are you will want some or all
%    of these if writing a dissertation that includes equations.
%  \usepackage{amsmath, amscd, amssymb, amsthm}

%% BONUS MATH
%  \usepackage{mathtools}

%% MARGIN REQUIREMENTS IN TITLES - Hyphenation in a Section Title does not always respect margin settings in Latex.  To force no hyphentation, uncomment the package below.
%  \usepackage[raggedright]{titlesec}

%% GRAPHICX - This is the standard package for
%    including graphics for latex/pdflatex.
\usepackage{scrextend}
\usepackage{pslatex}
\usepackage{graphicx}
\usepackage{amsmath}
\usepackage{bm}
\usepackage{amssymb}

%% CAPTION
% This overrides some of the ugliness in ucsd.cls and
% allows the text to be double-spaced while letting figures,
% tables, and footnotes to be single-spaced--all OGS requirements.
% NOTE: Must appear after graphics and ams math
\makeatletter
\gdef\@ptsize{2}% 12pt documents
\let\@currsize\normalsize
\makeatother
\usepackage{setspace}
\doublespace
\usepackage[font=small, width=0.9\textwidth]{caption}

%% SUBFIG - Use this to place multiple images in a
%    single figure.  Subfig will handle placement and
%    proper captioning (e.g. Figure 1.2(a))
% \usepackage{subfig}

%% TIMES FONT - replacements for Computer Modern
%%   This package will replace the default font with a
%%   Times-Roman font with math support.
% \usepackage[T1]{fontenc}
% \usepackage{mathptmx}

%% INDEX
%   Uncomment the following two lines to create an index:
% \usepackage{makeidx}
% \makeindex
%   You will need to uncomment the \printindex line near the
%   bibliography to display the index.  Use the command
% \index{keyword}
%   within the text to create an entry in the index for keyword.
%   To compile a LaTeX document with an index the 'makeindex'
%   command will need to be run.  See the wiki for more details.

%% HYPERLINKS
%   To create a PDF with hyperlinks, you need to include the hyperref package.
%   THIS HAS TO BE THE LAST PACKAGE INCLUDED!
%   Note that the options plainpages=false and pdfpagelabels exist
%   to fix indexing associated with having both (ii) and (2) as pages.
%   Also, all links must be black according to OGS.
%   See: http://www.tex.ac.uk/cgi-bin/texfaq2html?label=hyperdupdest
%   Note: This may not work correctly with all DVI viewers (i.e. Yap breaks).
%   NOTE: hyperref will NOT work in draft mode, as noted above.
% \usepackage[colorlinks=true, pdfstartview=FitV,
%             linkcolor=black, citecolor=black,
%             urlcolor=black, plainpages=false,
%             pdfpagelabels]{hyperref}
% \hypersetup{ pdfauthor = {Your Name Here},
%              pdftitle = {The Title of The Dissertation},
%              pdfkeywords = {Keywords for Searching},
%              pdfcreator = {pdfLaTeX with hyperref package},
%              pdfproducer = {pdfLaTeX} }
% \urlstyle{same}
% \usepackage{bookmark}


%% CITATIONS
% Sets citation format
% and fixes up citations madness
\usepackage{microtype}  % avoids citations that hang into the margin


%% FOOTNOTE-MAGIC
% Enables footnotes in tables, re-referencing the same footnote multiple times.
\usepackage{footnote}
\makesavenoteenv{tabular}
\makesavenoteenv{table}


%% TABLE FORMATTING MADNESS
% Enable all sorts of fun table tricks
\usepackage{rotating}  % Enables the sideways environment (NCPW)
\usepackage{array}  % Enables "m" tabular environment http://ctan.org/pkg/array
\usepackage{booktabs}  % Enables \toprule  http://ctan.org/pkg/array

%%%%%%%%%%%%%%%%%%%%%%%%%%%%%%%%%%%%%%%%%%%%%%%%%%%%%%
% redefine the comma to allow breaks in inline math
\makeatletter
\def\old@comma{,}
\catcode`\,=13
\def,{%
  \ifmmode%
    \old@comma\discretionary{}{}{}%
  \else%
    \old@comma%
  \fi%
}
\makeatother
%%%%%%%%%%%%%%%%%%%%%%%%%%%%%%%%%%%%%%%%%%%%%%%%%%%%%%


\begin{document}

%% FRONT MATTER
%
%  All of the front matter.
%  This includes the title, degree, dedication, vita, abstract, etc..
%  Modify the file template_frontmatter.tex to change these pages.
% %
%
% UCSD Doctoral Dissertation Template
% -----------------------------------
% http://ucsd-thesis.googlecode.com
%
%


%% REQUIRED FIELDS -- Replace with the values appropriate to you

% No symbols, formulas, superscripts, or Greek letters are allowed
% in your title.
\title{Periodicity Transforms and their Applications to Digital Audio}

\author{Jacob Sundstrom}
\degreeyear{\the\year}

% Master's Degree theses will NOT be formatted properly with this file.
\degreetitle{Doctor of Philosophy}

\field{Music}
\specialization{Computer Music}  % If you have a specialization, add it here

\chair{Professor Miller Puckette}
% Uncomment the next line iff you have a Co-Chair
% \cochair{Professor Cochair Semimaster}
%
% Or, uncomment the next line iff you have two equal Co-Chairs.
%\cochairs{Professor Chair Masterish}{Professor Chair Masterish}

%  The rest of the committee members  must be alphabetized by last name.
\othermembers{
Professor Shahrokh Yadegari\\
Professor Anthony Burr\\
Professor Clinton Tolley\\
}
\numberofmembers{4} % |chair| + |cochair| + |othermembers|


%% START THE FRONTMATTER
%
\begin{frontmatter}

%% TITLE PAGES
%
%  This command generates the title, copyright, and signature pages.
%
\makefrontmatter

%% DEDICATION
%
%  You have three choices here:
%    1. Use the ``dedication'' environment.
%       Put in the text you want, and everything will be formated for
%       you. You'll get a perfectly respectable dedication page.
%
%
%    2. Use the ``mydedication'' environment.  If you don't like the
%       formatting of option 1, use this environment and format things
%       however you wish.
%
%    3. If you don't want a dedication, it's not required.
%
%
\begin{dedication}
    To two, the loneliest number since the number one.
\end{dedication}


% \begin{mydedication} % You are responsible for formatting here.
%   \vspace{1in}
%   \begin{flushleft}
% 	To me.
%   \end{flushleft}
%
%   \vspace{2in}
%   \begin{center}
% 	And you.
%   \end{center}
%
%   \vspace{2in}
%   \begin{flushright}
% 	Which equals us.
%   \end{flushright}
% \end{mydedication}



%% EPIGRAPH
%
%  The same choices that applied to the dedication apply here.
%
\begin{epigraph} % The style file will position the text for you.
  \emph{It is better to be roughly right than precisely wrong.}\\
  ---John Maynard Keynes
\end{epigraph}

% \begin{myepigraph} % You position the text yourself.
%   \vfil
%   \begin{center}
%     {\bf Think! It ain't illegal yet.}
%
% 	\emph{---George Clinton}
%   \end{center}
% \end{myepigraph}


%% SETUP THE TABLE OF CONTENTS
%
\tableofcontents
\listoffigures  % Comment if you don't have any figures
\listoftables   % Comment if you don't have any tables



%% ACKNOWLEDGEMENTS
%
%  While technically optional, you probably have someone to thank.
%  Also, a paragraph acknowledging all coauthors and publishers (if
%  you have any) is required in the acknowledgements page and as the
%  last paragraph of text at the end of each respective chapter. See
%  the OGS Formatting Manual for more information.
%
\begin{acknowledgements}
 Thanks to whoever deserves credit for Blacks Beach, Porters Pub, and
 every coffee shop in San Diego.

 Thanks also to hottubs.
\end{acknowledgements}


%% VITA
%
%  A brief vita is required in a doctoral thesis. See the OGS
%  Formatting Manual for more information.
%
\begin{vitapage}
\begin{vita}
  \item[2012] B.~A. in Music \emph{Honors with Distinction}, Minor in Philosophy, University of California, San Diego
  \item[2015] M.~M. in Music, University of Washington, Seattle
  \item[2022] Ph.~D. in Music, University of California, San Diego
\end{vita}
\begin{publications}
  \item Your Name, ``A Simple Proof Of The Riemann Hypothesis'', \emph{Annals of Math}, 314, 2007.
  \item Your Name, Euclid, ``There Are Lots Of Prime Numbers'', \emph{Journal of Primes}, 1, 300 B.C.
\end{publications}
\end{vitapage}


%% ABSTRACT
%
%  Doctoral dissertation abstracts should not exceed 350 words.
%   The abstract may continue to a second page if necessary.
%
\begin{abstract}
  This dissertation will be abstract.
\end{abstract}


\end{frontmatter}


%% DISSERTATION
\chapter{Introduction}

In \cite{sethares1999periodicity}, W. A. Sethares and T. W. Staley introduced the concept of the ``periodicity transform" (PT) whereby a signal can be decomposed into periodic basis vectors by projection onto periodic subspaces. The PT excels in situations where the time series is described best in terms of period rather than frequency. This technique and its variations have been applied to astromonical data (\cite{buccheri1985time}), machine vibration (\cite{muresan2003orthogonal}), gene sequencing (\cite{arora2007detection}), and musical rhythms (\cite{sethares2001meter}); however, its applcation to time-domain audio signals directly has been lacking. Since the basis vectors calculated with the PT often have overlapping spectra, it seems a natural extention to apply the PT to both analysis and synthesis of audio, and especially to musical audio. In doing so, a short-time periodicity transform was developed in order to capture the changing periodic nature of musical signals and is applied to analyze and resynthesize an excerpt of a well-known electronic work.


Since Sethares and Staley in 1999, the PT has been developed in multiple directions. This chapter concerns itself with the examining and presenting existing versions and their applicability to audio analysis and synthesis.

\section{Nonorthogonal Periodic Decomposition}
Sethares and Staley

Original periodicity transform

Basis vectors are not orthogonal.

\section{Orthogonal Periodic Decomposition}

The orthogonal, exactly periodic decomposition of Muresan and Parks is based on the algorithm presented in \cite{wise1976maximum}.

In \cite{muresan2003orthogonal}, the process of orthogonal projection without constructing the orthogonal subspaces themselves is introduced. Let $x_p^\perp = \pi(x, P_p)^\perp$ be the orthogonal projection of $x$ onto subspace $P_p$.

To make $x_p \rightarrow x_p^\perp$, get the factors of $p$ so that $F_p = \{f : f|p, f \neq p\}$. Then project $x_p$ onto the subspaces corresponding with $F_p$, also subtracting out each of their factors as well from their own projections:
\begin{align}\label{intro:orthgonalProjection}
    \pi(x, P_p)^\perp = x_p - \sum_{f \in F_p} \pi(x_p, P_f)^\perp
\end{align}
where $x_p = \pi(x, P_p)$.

It turns out, however, that since $P_f \in P_p$ for $f|p$, it is not necessary to single out each factor of $p$ before subtracting out each projection. One can simply subtract the non-orthogonalized projection $\pi(x_p, P_f)$ from $x_p$ for $f \in F_p$ to get $x_p^\perp$. Depending on the order, however, one will get projections equal to 0 for periods $d$ if $d|f$ and $f$ has already been subtracted. For instance:
\begin{align*}
    x_{25} &= \pi(x, P_{25}) \\
    \pi(x_{25}, P_5) &= 0
\end{align*}

In order to avoid projecting and subtracting vectors that will equal zero, it is possible to project only at periods $p/f'$, where $f'$ is a prime factor. We then redefine Equation \eqref{intro:orthgonalProjection}:
\begin{align}
    \pi(x, P_p)^\perp = x_p - \sum_{f \in F'_p} \pi \bigg( x_p, P_{\frac{p}{f}} \bigg)
\end{align}
where $F'_p = F_p \cap \mathbb{P}$ and $\mathbb{P}$ is the set of all prime numbers. A proof of this is given in Appendix XXX.


\section{Ramanujan Periodic Decomposition}

Vaidyanathan and Tenneti

Struggles hardcore with non-integer periods. Suggestion when doing machine vibration is to adjust the samplerate so that the period aligns to an factor of the samplerate. This begs the question why one would have the need to search for a period one already knows is there. That is, without knowing the period a priori it would be impossible to adjust the samplerate accordingly.

\chapter{Periodic Detection}\label{chapter:detection}

As noted in XXX, XXX, and XXX, it is possible to frame the problem as a convex optimization program where the reconstruction error is minimized. However, it is important to note that the aim of this dissertation is to examine the application of the periodicity transform to audio processing (encoding to an extent). This means that there is a set of qualitative criteria that we would do well to adhere to. Specifically, the ``atoms'' into which a signal is decomposed must be ``meaningful'' in the sense that manipulating the atoms individually results in something that resembles ``atomic processes'' themselves. In other words, the idea is qualitative \emph{coherence} versus \emph{incoherence} of the atoms themselves. This will become increasingly clear in the remaining chapters.

This is essentially extending the idea of linearity into the qualitative domain.

The atoms are the periodic bases that describe the signal according to some qualitiative criteria which may or may not comport with the exact solution of a complete power minimization of the input signal.

This means that although the problem is framed as a power minimization problem, minimization qua minimization is not necessarily the end goal here. We are less concered with the encoding of information so as to create a perfect reconstruction of the input with a set of atoms as with the manipulation of the information after decomposition so that the combined result is qualitatively coherent with the original signal and underlying proceses.

The words ``bases'' and ``atoms'' are often used interchangebly in this text, although, in general, ``atoms`` refer to a single $p$-periodic component of length $p$ and ``bases'' refers to those atoms periodically extended to $N$, the length of the input signal.


Therefore, the primary concern is with retrieving an appropriate set of periodic atoms that describe the input signal to some tolerance such that the period $p$ of the base is such that $p \leq p_{\text{max}} < N$. These atoms must further be constrained by the qualitative condition that they must be "listenable components" in that when periodically extended, they sound as if they are a component of the sound that is analyzed. This will become clear in Chapter \ref{chap:issues} when the difficulties of non-integer periods are explored. Briefly, it is possible to construct a periodic signal with virtually no residual using the Natural Bases by means of periodic functions that bear no resemblance, apart from their mathematical derivation, to the input.\footnote{See the sound installation \emph{same old person, new mistakes (didn't even notice)} in the author's catalog for an aesthetic exploitation of this phenomenon.}

The process in this chapter describes a way by which one may decompose an input signal into component periodic waveforms that are "listenable". The enumerated list below shows and briefly describes each step. We begin by letting $x$ be the input signal and we initialize $x_r = x$.
\begin{enumerate}
    \item\textbf{Detection:} Choose the best periodic projection, $x_{r_{_p}}$, for $x_r$. Keep a record of all best periods found so far in the set $Q_i$ for step $i$. The addition of this result to $Q$ is then checked against the \emph{stop function}, briefly described below.
    \item\textbf{Reconstruction:} Construct a basis matrix that spans the space of $x$ as described by periods $Q_i$ using the Natural Bases. Compute the reconstruction and the residual error by means of a convex program.
    \item\textbf{Stop Function:} Compute the value of a stop function, $s()$, that determines whether or not to continue processing. This function is arbitrary and may have arbitrary arguments.
    \item\textbf{Recursion:} If one continues processing, set $x_r = x - \sum_{q \in Q} x_q$ and repeat the process until $s()$ calls to stop.
\end{enumerate}
The fundamental difference between this and \cite{sethares1999periodicity} is the difference in computing the residual. That is, one can frame the residual computation as a quadratic program which allows one to alter the result of the Natural Basis to one that minimizes the reconstruction error.


%%%%%%%%%%%%%%%%%%%%%%%%%%%%%%%%%%%%%%%%%%%%%%%%%%%%%%%%%%%%%%%%%%%%%%%%%%%%%%%%
%%%%%%%%%%%%%%%%%%%%%%%%%%%%%%%%%%%%%%%%%%%%%%%%%%%%%%%%%%%%%%%%%%%%%%%%%%%%%%%%
\section{Detection}
Given an input, the first task is to derive what integer-valued period best describes the signal. Several methods have been described in the literature (LIST SOME REFS HERE) and has been the primary are of research on the topic of periodicity transforms. For the purposes of audio manipulation and processing not limited to encoding and reconstruction, some proved to be more useful than others. The method outlined here has shown, in practice, to provide the most useful results for audio processing and is essentially a combination of the iterative methods of \cite{sethares1999periodicity} and the underlying processes of \cite{muresan2003orthogonal}.

In general, period detection is a maximum likelihood estimator (MLE) where the power of some period $p$ is taken on all allowable $p$'s and the result with the maximum power is chosen. Let $P_x$ be the ordered collection of periodic powers in $x$, calculated using one of the methods presented below. To choose the most likely integer period that describes the data, one simply does:
\begin{align}
    \hat{p}_{i} = argmax(P_x)
\end{align}
where $\hat{p}_{i}$ is the most likley period in $x$ at step $i$. Notice that due to the nature of this process, we select a \emph{single} best period and choose additional periods during recusion and as a function of the stop function.

    \subsection{Non-orthogonalized Selection}
    The non-orthogonal method of Sethares and Staley's $M$-best $\gamma$ has proven to be useful in recovering component periods although it is prone to noise and miscategorization of hidden periods. Notably, it is often the case that common divisors of hidden periods are chosen in the MLE since the combined power of each component sharing a common divisor is often large due to the construction of Equation \eqref{eq:intro:sethares:periodicNormGamma}. That is, the columns of the $P_p$ matrix are not orthogonal to each other. It is therefore possible, indeed likley, that during reconstruction, two projections $x_{p_0}$ and $x_{p_1}$ will both contribute to a common divisor period, $d$. Moreover, the $M$-best $\gamma$ periodic power equation biases toward smaller periods, increasing the likelihood that $d$ would be chosen in the example above.

    In order to calculate $P_x$, one must project $x$ independently onto all possible $p$'s, then take the periodic power of each $x_p$, selecting the one with the highest power:
    \begin{align}
        P_{x_{p}} = \frac{||x_p||^2} {\sqrt{Np}}
    \end{align}
    The projection itself is performed by using Equation \eqref{eq:intro:setharesProjection}.
    \footnote{When projecting, it is often easist (and fastest) to rearrange the single row of input $S$ into a rectuangular matrix, zero-padded as necessary. Suppose $x$ is of length $N$ and we are searching for the projection of period $p$ where $N \neq pm$ for an integer $m$. We first extend $x$ by zero-padding the appropriate number of samples until $N = pm$ and reshape $x$ into a rectangular matrix:
    \begin{align}
        &\overline{x} = \text{zp}(x, N - (N \text{ mod } p)) \label{eq:detection:overlineS} \\
        &\overline{x}^{\text{ } 1 \times N} \rightarrow \overline{x}^{\text{ } m \times p} \label{eq:detection:overlineSReshaped}
    \end{align}
    where $m$ is an integer such that $N = mp$. This is equivalent projecting onto the $\bm{B}_p$ subspace and omitting the zeros in the $\delta^s_p$ functions. As such, one can the sum the matrix column-wise and divide each entry by the number of rows except for the final $N - (N \text{ mod } p)$ columns, which are divided by the number of rows minus 1. Notice that this simply averages the values in the columns which correspond to a particular $\delta^s_p$ vector, as \cite{sethares1999periodicity} does.}

    % Each basis matrix of period $p$ has $\varphi(p)$ columns that are unique to $P_p$. This means that it is possible to take an orthogonal version of $P_p$ by taking only the first $\varphi(p)$ columns of $P_p$ and setting the rest of the columns to zero. Unfortunately, doing this directly results in projections which have $p - \varphi(p)$ zeros in every periodic repetition. This is untenable for a vareity of reasons.

    % In order to distinguish a set of basis vectors constructed using the Natural Basis, let $\bm{B}_p$ be a $N \times p$ matrix whose columns are periodic with period $p$ so that
    % \begin{align}
    %     \bm{B}_p &= \begin{bmatrix}
    %             \delta^0_p(N) &
    %             \delta^1_p(N) &
    %             \hdots &
    %             \delta^{p-1}_p(N) &
    %         \end{bmatrix}^{T}
    % \end{align}
    % where $\delta^s_p(N)$ is Eq. \ref{eq:naturalBasisVector}.

        %
        %
        % \subsubsection{Windowed Projection}
        % If $N - (N \text{ mod } p) \neq 0$, the projection will contain artifacts at values around $N - (N \text{ mod } p)$ since by cutting the $p$-periodic component at a partial repetition, one introduces a step function. This effect is exacerbated as $p$ approaches $N$. \cite{muresan2003orthogonal} suggests truncating the input $x$ to be a multiple of $p$ but this is not always desirable or possible if $N$ is small relative $p$. We can mitigate the the artifacts of $N$ not being an integer multiple of the proposed period $p$ by windowing the input and dividing by the appropriate value which will not be the same for each column. This is an application of a weighted mean to a window where the mean is each element multiplied by its weight divided by the sum of the weights:
        % \begin{align}
        %     \overline{m}_{xw} = \frac{\sum_{i=0}^{N-1} x[n]w[n]} {\sum_{n=0}^{N-1} w[n] }
        % \end{align}
        % where $x$ is a sequence of length $N$ and $w$ is a 1-normalized window function also of length $N$. Notice that this works only if $\sum_{i=0}^{N-1} w[n] \geq 1$ which is why one must normalize. By normalizing, we ensure that the sum will always be greater than 1 and that the proportion of the weights is maintained.
        %
        % In order to derive the values by which we outght to divide by when using a window, we take the window function $w$ and apply the same process as in Equations \eqref{eq:detection:overlineS} and \eqref{eq:detection:overlineSReshaped}, then sum downwards. This results in a vector of length $p$ whose values are used to divide, column-wise, by the same length vector resulting from the column-wise summation of $\overline{S}^{\text{ } m \times p}$. The windowing in essense more heavily ``weights'' the center of the signal where a complete period is more likley to be present and unweights the edges where partial repetitions may be present. This is easily expanded for a no window situation, where we can essntially say that the window is a boxcar (all 1's). Moreover, the application of a window does not exacerbate the effect of an analysis and/or recovery if $N$ is not a large multiple of the contained periods.
        %
        % This windowing method has the additional benefit of offsetting some of the effects if the signal is enveloped. Suppose $S$ is a signal which is the composite of two sawtooth waves with periods $p_0$ and $p_1$ with a ramp envelope from $0.5$ to $1$. Here $N$ is not an integer multiple of either $p_0$, $p_1$, nor $p_0 p_1$. Figure XXX, top, shows the composite signal signal; the envelope is plainly visible. In the lower portion, the solid lines shows the recovered waveforms without windowing (or with a boxcar window) while the dashed lines shows the recovered waveforms when applying a Hann window.
        %
        % For the rest of this document, we will always include a window function where appropriate and denote it $w$. Thinking specifically of the sections where we project with $A$ and stuff. Probably. Show that this actually gives a better residual.

    \subsection{Orthogonal Detection}
    It is often far preferable to detect the periodic components \emph{orthogonally}; that is, using the method of \cite{muresan2003orthogonal}, described in Section \ref{section:intro:orthgonalDecomp}. This is mainly due to the aforementioned fact that for the method of Sethares and Staley above, energies become concentrated in common divisors of the actual periodic components. We follow \cite{muresan2003orthogonal} in avoiding the direct calculation of the orthogonal subspace $P_{p}^{\perp}$ by using Equation \eqref{eq:intro:orthgonalProjection}. We can see that the distribution of powers is much more like what we would expect versus the method of \cite{sethares1999periodicity}:

    FIGURE HERE

    The computational advantage here is that one need only project \emph{once} after $p_{i}$ is found for a given iteration. An important note here is that the orthogonal method of Muresan and Parks \emph{always} truncates the input signal to an integer multiple of the proposed period $p$. This, however, has not been shown to be problematic in practice.

%%%%%%%%%%%%%%%%%%%%%%%%%%%%%%%%%%%%%%%%%%%%%%%%%%%%%%%%%%%%%%%%%%%%%%%%%%%%%%%%
%%%%%%%%%%%%%%%%%%%%%%%%%%%%%%%%%%%%%%%%%%%%%%%%%%%%%%%%%%%%%%%%%%%%%%%%%%%%%%%%
\section{Reconstruction}
Let $Q_i$ be the set of periods found in $x$ by means of a projection method at step $i$. (Note that $|Q_i| = i + 1$ at the end of the detection stage in step $i$) We now ask ourselves: what is the best representation of $x$ using only periods in $Q_i$? For convience, we will often denote $Q_i$ as simply $Q$, though a particular $i$ is implied.

In \cite{sethares1999periodicity}, the reconstruction is computed by simply summing the projections $x_q$ for $q \in Q$. That is:
\begin{align}
    \hat{x} &= \sum_{q \in Q} x_q
\end{align}

In other literature, reconstruction of $x$ is performed similarly by solving
\begin{align} \label{eq:detection:convex}
    \bm{A}x &= b
\end{align}
then doing
\begin{align}
    \hat{x} = \bm{D}_{Q}^{T} b
\end{align}
where $\bm{A}$ is a the covariance matrix of $\bm{D}_Q$ which is a matrix of basis vectors that spans the space of $x$ using periods $Q$ with linearly independent columns. In most cases, $D_Q$ will not be a square matrix since its row dimension is $N$. For the purposes of this dissertation, this is the primary method of reconstruction. The question now becomes: how must one construct $\bm{D}_Q$?

    \subsection{Construction of $\bm{D}_Q$}\label{section:detection:D_Q}
    Let $\bm{D}_Q$ be a ``basis matrix'', or a matrix of linearly independent columns which span the space of the sequence $x$ with periods $q \in Q$ of shape $N \times c_{\bm{D}_Q}$. The column dimensionality, $c_{\bm{D}_Q}$, is defined as
    \begin{align}
        R &= \bigcup_{q_i \in Q} \text{factors}(q_i) \label{eq:A:R} \\
        c_{\bm{D}_Q} &= \sum_{R_i \in R} \varphi(R_i) \label{eq:A:c_A}
    \end{align}
    where $R$ is the union of all the factors of the periods in $Q$, $\text{factors}(n)$ is a function which returns the set of factors of an integer $n$, and $\varphi(n)$ is Euler's totient function. The columns of $\bm{D}_Q$ itself are the Natural Basis of period $p$ as defined in Eq. \ref{eq:naturalBasis}:
    \begin{align}\label{eq:detection:overdeterminedDQ}
        D_Q = \begin{bmatrix}
            \bm{B}_{q_{0}} & \bm{B}_{q_{1}} & \hdots & \bm{B}_{q_{i}}
        \end{bmatrix}^{T}
    \end{align}

    Recall two things at this point: that the subspace $P_p$ for a given period $p$ is a matrix of shape $N \times p$ and that each $P_p$ is not orthogonal to others. This means that $D_Q$ as constructed in Equation \eqref{eq:detection:overdeterminedDQ} is overdetermined. Therefore, in order to avoid introducing additional power at the period $d$ for $d|p$ and have linearly independent columns, one must remove columns from each $\bm{P}_p$ accordingly. But what columns to remove?

    According to Lemma XX in \ref{tenneti2016unified}, a Natural Basis matrix that is a full rank consists of keeping the first $\varphi(p)$ columns of $\bm{P}_p$ for all integers betwen 1 and $p$. A complete and not overdetermined dictioary of Natural Bases for periods 1 to $p_{\text{max}}$ therefore consists of keeping the first $\varphi(p)$ columns of each $\bm{B}_p$ for $1 \leq p \leq p_{\text{max}}$. This is essentially saying that for a periodic subspace of period $p$, there are $\varphi(p)$ degrees of independence between $p$ and all other possible $p$'s. In this case, however, it is not true that we want $\bm{D}_Q$ to contain all periods between 1 and $p_{\text{max}}$ but rather only those periods $q \in Q$.
     Nonetheless, this gives an idication of how to proceed of which there are two ways: (1) we can construct $\bm{D}_Q$ so that we separate out every factor which appears in two or more $q \in Q$ and take care to remove that factor from $\bm{B}_p$ by removing columns or, (2) we can construct $\bm{D}_Q$ by keeping track of what periods have been added so far, taking care to not add a factor twice. Although both methods result in a matrix with spans the same space and are functionally equuivalent, the coefficients returned from the convex program have different meanings. The importance of this will become clear in subsequent sections.
    %%%%% ^^^^ should probably prove that these are equivalent?!?! ^^^^^^^^

        \subsubsection{Separating out common factors in $\bm{D}_Q$}\label{detection:A:separateFactors}
        For this method, we create a separate subspace for a period $d$ for any $d$ such that for any two periods $q_n, q_m \in Q$, $d \in \text{factors}(q_n)$ and $d \in \text{factors}(q_m)$. In other words, suppose $Q = \{8, 10\}$. There are two common factors, 1 and 2. Therefore, the 1-space and 2-space (which itself is absent the 1-space it contains) are separated out into separate basis matricies. In this case, then,
        \begin{align*}
            D_Q = \begin{bmatrix}
                \bm{B}_{1} & \bm{B}_{2} & \bm{B}_{8} & \bm{B}_{10}
            \end{bmatrix}^{T}
        \end{align*}
        Notice that although 8 also contains a factor of 4 and 10 contains a factor of 5, there is no need to separate out these spaces since any energy in the 5-space will \emph{only} be contained in the 10-space. Likewise with the 4-space and 8-space.

        Begin by defining $R_i$:
        \begin{align}\label{eq:A:commonFactors:R}
            R_i &= \Bigg( \bigcup_{p, q \in Q_i} \text{factors}(p) \cap \text{factors}(q) \Bigg) \cup Q_i
        \end{align}
        Notice that $R_i$ contains all elements of $Q_i$ as well as any common factors between elements in $Q_i$. In accordance with the previous convention, we will remove the subscript $i$ for simplicity of notation. To find the column dimensionality, $c_{r}$, of the basis matrix $\bm{B}_r$ for $r \in R$:
        \begin{align}\label{eq:A:commonFactors:cp}
            c_{r} &= r - \sum_{d|r \in R \setminus r} \varphi(d)
        \end{align}
        Therefore, to constuct $\bm{D}_Q$, we simply concatenate the subspaces $\bm{B}_r$ for $r \in R$:
        \begin{align}\label{eq:A:commonFactors:A}
            \bm{D}_Q &= \begin{bmatrix}
                    \bm{B}_{r_0}^{c_{r_0}} &
                    \bm{B}_{r_1}^{c_{r_1}} &
                    \hdots &
                    \bm{B}_{r_n}^{c_{r_n}}
                \end{bmatrix}
                , \text{ } r \in R
        \end{align}
        where the superscript indicates the number of columns of $\bm{B}$ to keep. Using this method, it is not imporant which order we concatenate the subspaces but only that we do not doubly count a particular subspace.\footnote{In practice and in conjunction with the extraction method presented in Chapter \ref{chapter:extraction}, knowing the order of the columns and what part of what period they represent is of critical importance.} One can confirm the dimensionality by using Eq. \eqref{eq:A:c_A} and checking that $\sum_{i = 0}^{|R| - 1} c_{r_i} = c_{\bm{D}_Q}$ where $|R|$ is the cardinality of $R$.

        As an example, suppose $Q = \{8, 10, 15\}$. According to Eq. \eqref{eq:A:P}:
        \begin{align*}
            R &=
                \big(
                \text{ } ( \text{factors}(8) \cap \text{factors}(10) ) \text{ } \cup
                \text{ } ( \text{factors}(8) \cap \text{factors}(15) ) \text{ } \cup
                \text{ } ( \text{factors}(10) \cap \text{factors}(15) )
                \text{ } \big)
                \text{ } \cup \{8, 10, 15\} \\
            R &= \{1, 2, 5\} \text{ } \cup \text{ } \{8, 10, 15\} \\
            R &= \{1, 2, 5, 8, 10, 15\}
        \end{align*}
        Now we can calculate the dimensionality of an arbitrary $p \in P$ in any order:
        \begin{align*}
            c_1 &= 1 - \sum_{d|1 \in \{2, 5, 8, 10, 15\}} \varphi(d) = 1 \\
            c_2 &= 2 - \sum_{d|2 \in \{1, 5, 8, 10, 15\}} \varphi(d) = 1 \\
            \vdots \\
            c_{10} &= 10 - \sum_{d|10 \in \{1, 2, 5, 8, 15\}} \varphi(d) = 10 - (1 + 1 + 4) = 4 \\
            \text{etc...}
        \end{align*}
        We see then that in this case,
        \begin{align*}
            D_Q = \begin{bmatrix}
                \bm{B}_{1}^{1} &
                \bm{B}_{2}^{1} &
                \bm{B}_{5}^{4} &
                \bm{B}_{8}^{6} &
                \bm{B}_{10}^{4} &
                \bm{B}_{15}^{10} &
            \end{bmatrix}^{T}
        \end{align*}

    \subsubsection{Progressive concatenation of $\bm{D}_Q$}\label{detection:A:progressiveConcatenation}
    What if one wants to construct $\bm{D}_Q$ in such a way as to \emph{only} include subspaces which are constructed from the elements of $Q$? The advantages of this method is that it is not necessary to add subspaces of factors for the periods we are actually interested in and that it (somewhat) simplifies the process of computing the waveforms containing their subcomponents. However, it requires one to keep track of the order in which periods are added to the subspace. That is, each periodic component $q \in Q$ also contains the components $d|q$ and one must construct the basis matrix with this in mind, taking care to remove the component from a subspace if it is already present.

    To do so, we exploit the fact that $n = \sum_{f|n} \varphi(f)$. Let $c_{\bm{D}_{Q_i}}$ denote the column dimensionality of $\bm{D}_Q$ at step $i$. We also redefine Eq. \eqref{eq:A:R}:
    \begin{align}\label{eq:A:onlyPeriods:R}
        R_i = \bigcup^i_{q_j \in Q_i} f(q_j)
    \end{align}
    Given the above, the total column dimensionality $c_{\bm{D}_{Q_i}}$ at step $i$ is:
    \begin{align*}
        c_{\bm{D}_{Q_i}} = \sum_{r \in R_i} \varphi(r)
    \end{align*}
    Therefore, the column dimensionality for any $q$ at step $i$ is:
    \begin{align}\label{eq:A:noFactors:cp}
        c_{q_i} =
            \begin{cases}
                q_i, & \text{if } i = 0 \\
                c_{\bm{D}_{Q_i}} - c_{\bm{D}_{Q_{i-1}}}, & \text{otherwise}
            \end{cases}
    \end{align}
    Again, we will simplify the notation going forwrd by omitting the subscript $i$. We construct $\bm{D}_Q$ as:
    \begin{align*}
        \bm{D}_Q &= \begin{bmatrix}
                \bm{B}_{q_0}^{c_{q_0}} &
                \bm{B}_{q_1}^{c_{q_1}} &
                \hdots &
                \bm{B}_{q_n}^{c_{q_n}}
            \end{bmatrix}
            , \text{ } \forall q \in Q
    \end{align*}
    In doing so, we ensure that for every subsequent subspace in $\bm{D}_Q$ is orthogonal to all existing subspaces in $\bm{D}_Q$.

    Returning to the example in the previous section, we see that
    \begin{align*}
        c_{q_0} &= c_{8} = 8 - 0 = 8 \\
        c_{q_1} &= c_{10} = 16 - 8 = 8 \\
        c_{q_2} &= c_{15} = 26 - 16 = 10 \\
    \end{align*}
    Therefore:
    \begin{align*}
        \bm{D}_Q = \begin{bmatrix}
            \bm{B}_{8}^{8} & \bm{B}_{10}^{8} & \bm{B}_{15}^{10}
        \end{bmatrix}
    \end{align*}
    The total column dimensionality can again can be double-checked using Eq. \eqref{eq:A:c_A}.

    \subsection{Approximation of $x$ using $\bm{D}_Q$}
    Once $\bm{D}_Q$ has been appropriately constructed, we can compute $\bm{A}$ as the covariance matrix of $\bm{D}_Q$:
    \begin{align}
        \bm{A} = \bm{D}_Q \bm{D}_Q^T
    \end{align}
    One then simply puts $\bm{A}$ in the convex program and solves for the reconstruction, $\hat{x}$, with matrix multiplication:
    \begin{align}
        \bm{A}x &= b \label{eq:detection:reconstruction:coefficients} \\
        \hat{x} &= \bm{A} b^{T} \label{eq:detection:reconstruction:xHat}
    \end{align}

    Notice that due to the reconstitution of $\bm{D}_Q$ on each iteration, which in turn recomputes $\bm{A}$, the resulting basis vectors, when retrived using the method described in Chapter \ref{chapter:extraction} can change so as to minimize $x - \hat{x}$. This possible change of the bases on each iteration is also part of the motation of the stop function, briefly described in \ref{section:detection:stopFunction} below. Further, reconstituting $\bm{D}_Q$ each iteration means that the coefficients $b$ and reconstruction $\hat{x}$ is identical to projecting onto the subspaces $P_q$ for $q \in Q$ simultaneously as is possible with orthogonal dictionaries of each component and its factors.

%%%%%%%%%%%%%%%%%%%%%%%%%%%%%%%%%%%%%%%%%%%%%%%%%%%%%%%%%%%%%%%%%%%%%%%%%%%%%%%%
%%%%%%%%%%%%%%%%%%%%%%%%%%%%%%%%%%%%%%%%%%%%%%%%%%%%%%%%%%%%%%%%%%%%%%%%%%%%%%%%
\section{Stop Function}\label{section:detection:stopFunction}
The selection of a stop function is critical for periodic decomposition success if one wishes to pay respect to the qualitiative criteria at the start of this chapter. That is, any formulation of the periodicity transform will invariably detect a peroidic waveform of some period $p$ with an input of Gaussian noise. In other words, just because the detection step returns a ``best'' period does not mean that that period has any correspondence to the actual content of the signal and would be of no interest in audio processing. The ``best'' period in the residual may contribute to minimizing the reconstruction error but this is a fundamentally different prospect than retriving the integer periods which best describe the signal.

It is important to note that the stop function is evaluated \emph{before} the detection step on each iteration. This topic will be discussed further in Section \ref{section:audioEffects:stopFunction} but suffice it to say, for now, that it is almost always a necessary condition of the detection stage as described in this dissertation.
%%%%%%%%%%%%%%%%%%%%%%%%%%%%%%%%%%%%%%%%%%%%%%%%%%%%%%%%%%%%%%%%%%%%%%%%%%%%%%%%
%%%%%%%%%%%%%%%%%%%%%%%%%%%%%%%%%%%%%%%%%%%%%%%%%%%%%%%%%%%%%%%%%%%%%%%%%%%%%%%%
\section{Recursion}\label{section:detection:recursion}
This step hardly needs explaination: if the criteria in the stop function is met, cease processing. Otherwise, continue to process, adding to the total set of periods found.


% Define $P$:
% \begin{align*}
%     P &= \Bigg( \bigcup_{\forall p, q \in Q} f(p) \cap f(q) \Bigg) \cup Q
% \end{align*}
% i.e. the union of the intersections of the factors of all the elements in $Q$.
%
% To find the column dimensionality, $c_{p}$, of the basis matrix $\bm{B_p}$ for $p \in P$:
% \begin{align*}
%     % c_{p} &= p - \sum_{f : f(p) \cap P \setminus p} \varphi(f)
%     c_{p} &= p - \sum_{d|p \in P \setminus p} \varphi(d)
% \end{align*}
%
% %% If N is the columns
% % We then define $\bm{A}$ as:
% % \begin{align*}
% %     \bm{A} &= \begin{bmatrix}
% %             \bm{B}_{p_0} \\
% %             \bm{B}_{p_1} \\
% %             \vdots \\
% %             \bm{B}_{p_n} \\
% %         \end{bmatrix}
% %         , \text{ } \forall p \in P
% % \end{align*}
% % where there are $n$ elements in $\bm{P}$.
%
% To find the total column dimensionality, $c_{\boldsymbol{A}}$ of the matrix $\bm{A}$:
% \begin{align*}
%     c_{\bm{A}} &= \sum_{p \in P} c_p
% \end{align*}

% In this way, we can be assured that not only are the basis vectors in $\bm{A}$ linearly independent, but also that all traces of some period $q$ are removed from the residual signal $h$.

% \subsection{This also works}
% We can also get a basis matrix for the periods in $q$ by using the same equation as above that uses $varphi(n)$ by creating $\bm{A}$ along the way and keeping track of which periods (and factors of the periods) we've added along the way.
%
% Let $F = \emptyset$. To find the column dimensionality, $c_q$, for $q \in Q$:
% \begin{align}
%     c_q &= q - \sum_{d|q \in F} \varphi(d)
% \end{align}
%
% We then add elements to $F$ every time we add a basis matrix $\bm{B}_q$ to $\bm{A}$:
% \begin{align*}
%     F &= \{ F \cup f(q) \}_{\neq}
% \end{align*}
% i.e. every time we process a period $q$, add all the factors of $q$ to $F$, removing duplicates.
%
% Repeat this process for every $q \in Q$, adding to $F$ along the way. (This means that we need to keep track of which factors we've added along the way which makes it uglier in some ways, better in others.)
%
% Therefore, we then define $\bm{A}$ as:
% \begin{align*}
%     \bm{A} &= \begin{bmatrix}
%             \bm{B}_{q_0} &
%             \bm{B}_{q_1} &
%             \hdots &
%             \bm{B}_{q_n}
%         \end{bmatrix}
%         , \text{ } \forall q \in \bm{Q}
% \end{align*}
% where there are $n$ elements in $\bm{Q}$.
%
% \section{Not a real section but describing getting the real periods back}

\chapter{Extraction of Periodic Bases}\label{chap:extraction}

In contrast to other literature on the periodicity transform, where the primary and concern is the detection of periodicities and extraction of a single periodic waveform, we are concerned here with also hearing the basis vectors themselves.

We want to get the periodic bases out so that they are not necessarily exactly periodic, as is the concern when searching for the periods themselves. Suppose there is a signal, $x$, that is composed of of two signals, $x_0$ an $x_1$ that contain components at each of their factors down to 1 (i.e. they are not exactly periodic) and whose fundamental periods are not mututally prime. Let $C = \text{factors}(x_0) \cap \text{factors}(x_1)$ where $\text{factors}()$ is Equation \eqref{eq:getFactors}. We can see that the two signals share components at periods $c \in C$ and that in order to extract the periods, one must determine how much of each common component is contained in each signal.


% We need to create redistribution matricies in order to properly distribute the various factors of Q into all other possibilities. That is, suppose two periodic functions $f$ and $g$ of periods $p_f$ and $p_g$ have a common factor $d$. We can detect the $d$-component of both $f$ and $g$ independently but when it comes time to reconstruct, how do we know how much of the $d$-component goes into f and how much of the $d$-component goes into $g$? We can again solve this by formulating a convex program and solving in the standard format or by least-squares. In this case, however, we will concatenate the period coefficients, zero-padded if necesssary, as the input to the convex program.

In order to solve this, we also set up a convex program whose input vector is the concatenation of the periods $q \in Q$ using their respecitve values in $y$ (output of Equation \eqref{eq:detection:convex}), zero padding as necessary so that each period is ``filled out''. Let this concatenation of periods be $w$.

\section{Construction of $w$}
An important note here is the order in which the periodic vectors were placed in $\bm{M}_Q$ since this determintes their corresponding coefficeints in the output of the convex program, $y$. Thus, the method by which $\bm{M}_Q$ was constructed is of importance and determines the method by which we construct $w$.

    \subsection{Separate Common Factors}
    The method here corresponds to the construction method in \ref{detection:A:separateFactors}. We know from Equation \eqref{eq:A:commonFactors:cp} that each component in $P$ as defined in Equation \eqref{eq:A:commonFactors:P} contains exactly $c_{p}$ columns which correspond to exactly $c_{p}$ coefficients in the output vector $y$. The position of these coefficients in $y$ corresponds to the order in which they were added to create $\bm{M}_Q}$.
    We then know that for $p_0$ the coefficeints that correspond to this period in $y$ are the first $c_{p_0}$ values. Likewise, for $p_1$, the coefficeints in $y$ are the next $c_{p_1}$ values. Generalizing, we see that the values in $y$ that correspond to the $i$th period from $P$ are:
    \begin{align}
        w_{p_i}^{*} &= y_j \text{, } \alpha_{i-1} \leq j \leq \alpha_{i} \\
        \alpha_{i} &= \sum_{k = 0}^{i} c_{p_k}
    \end{align}
    Notice that $w_{p_i}^{*}$ is not guaranteed to be of legnth $p_i$.\footnote{In fact, in the case of separating out factors to form $\bm{M}_Q$, it will never be.} We therefore zero-pad $w_{p_i}^{*}$ to get $w_{p_i}$:
    \begin{align}
        w_{p_i} = \text{zp}(w_{p_i}^{*}, p_i)
    \end{align}
    where $\text{zp}(x, N)$ is a function which concatenates zeros to the end of $x$ until $|x| = N$. $w$ is then formed by concatenation:
    \begin{align}
        w = \begin{bmatrix}
            w_{p_0} & w_{p_1} & w_{p_2} & \hdots & w_{p_n}
        \end{bmatrix}
    \end{align}

    \subsection{Progressive Concatenation}
    The method here corresponds to the construction method in \ref{detection:A:progressiveConcatenation}.


\section{Creating redistribution matricies}
Once the periods have been concatenated, we then have to construct $\bm{A}$ for input into the convex program. In contrast to $\bm{A}$ in Chapter \ref{chapter:detection}, 







% Define $P$:
% \begin{align*}
%     P &= \Bigg( \bigcup_{\forall p, q \in Q} f(p) \cap f(q) \Bigg) \cup Q
% \end{align*}
% i.e. the union of the intersections of the factors of all the elements in $Q$.
%
% To find the column dimensionality, $c_{p}$, of the basis matrix $\bm{B_p}$ for $p \in P$:
% \begin{align*}
%     % c_{p} &= p - \sum_{f : f(p) \cap P \setminus p} \varphi(f)
%     c_{p} &= p - \sum_{d|p \in P \setminus p} \varphi(d)
% \end{align*}
%
% %% If N is the columns
% % We then define $\bm{A}$ as:
% % \begin{align*}
% %     \bm{A} &= \begin{bmatrix}
% %             \bm{B}_{p_0} \\
% %             \bm{B}_{p_1} \\
% %             \vdots \\
% %             \bm{B}_{p_n} \\
% %         \end{bmatrix}
% %         , \text{ } \forall p \in P
% % \end{align*}
% % where there are $n$ elements in $\bm{P}$.
%
% To find the total column dimensionality, $c_{\boldsymbol{A}}$ of the matrix $\bm{A}$:
% \begin{align*}
%     c_{\bm{A}} &= \sum_{p \in P} c_p
% \end{align*}

% In this way, we can be assured that not only are the basis vectors in $\bm{A}$ linearly independent, but also that all traces of some period $q$ are removed from the residual signal $h$.

% \subsection{This also works}
% We can also get a basis matrix for the periods in $q$ by using the same equation as above that uses $varphi(n)$ by creating $\bm{A}$ along the way and keeping track of which periods (and factors of the periods) we've added along the way.
%
% Let $F = \emptyset$. To find the column dimensionality, $c_q$, for $q \in Q$:
% \begin{align}
%     c_q &= q - \sum_{d|q \in F} \varphi(d)
% \end{align}
%
% We then add elements to $F$ every time we add a basis matrix $\bm{B}_q$ to $\bm{A}$:
% \begin{align*}
%     F &= \{ F \cup f(q) \}_{\neq}
% \end{align*}
% i.e. every time we process a period $q$, add all the factors of $q$ to $F$, removing duplicates.
%
% Repeat this process for every $q \in Q$, adding to $F$ along the way. (This means that we need to keep track of which factors we've added along the way which makes it uglier in some ways, better in others.)
%
% Therefore, we then define $\bm{A}$ as:
% \begin{align*}
%     \bm{A} &= \begin{bmatrix}
%             \bm{B}_{q_0} &
%             \bm{B}_{q_1} &
%             \hdots &
%             \bm{B}_{q_n}
%         \end{bmatrix}
%         , \text{ } \forall q \in \bm{Q}
% \end{align*}
% where there are $n$ elements in $\bm{Q}$.
%
% \section{Not a real section but describing getting the real periods back}


%% END MATTER


% \printindex %% Uncomment to display the index
% \nocite{}  %% Put any references that you want to include in the bib
%               but haven't cited in the braces.
\bibliographystyle{alpha}  %% This is just my personal favorite style.
%                              There are many others.
%\setlength{\bibleftmargin}{0.25in}  % indent each item
%\setlength{\bibindent}{-\bibleftmargin}  % unindent the first line
%\def\baselinestretch{1.0}  % force single spacing
%\setlength{\bibitemsep}{0.16in}  % add extra space between items
\bibliography{template}  %% This looks for the bibliography in template.bib
%                          which should be formatted as a bibtex file.
%                          and needs to be separately compiled into a bbl file.
\singlespace  %to force bibilography environment to use single spacing for each entry
              %double spacing between entries remains
\end{document}
