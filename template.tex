%
%
% UCSD Doctoral Dissertation Template
% -----------------------------------
% https://github.com/ucsd-thesis/ucsd-thesis
%
%
% ----------------------------------------------------------------------
% WARNING:
%
%   This template has not endorced by OGS or any other official entity.
%   The official formatting guide can be obtained from OGS.
%   It can be found on the web here:
%   http://grad.ucsd.edu/_files/academic-affairs/Dissertations_Theses_Formatting_Manual.pdf
%
%   No guaranty is made that this LaTeX class conforms to the official UCSD guidelines.
%   Make sure that you check the final document against the Formatting Manual.
%
%   That being said, this class has been routinely used for successful
%   publication of doctoral theses.
%
%   The ucsd.cls class files are only valid for doctoral dissertations.
%
%
% ----------------------------------------------------------------------
% GETTING STARTED:
%
%   Lots of information can be found on the project wiki:
%   http://code.google.com/p/ucsd-thesis/wiki/GettingStarted
%
%
%   To make a pdf from this template use the command:
%     pdflatex template
%
%
%   To get started please read the comments in this template file
%   and make changes as appropriate.
%
%   If you successfully submit a thesis with this package please let us
%   know.
%
%
% ----------------------------------------------------------------------
% KNOWN ISSUES:
%
%   Currently only the 12pt size conforms to the UCSD requirements.
%   The 10pt and 11pt options make the footnote fonts too small.
%
%
% ----------------------------------------------------------------------
% HELP/CONTACT:
%
%   If you need help try the ucsd-thesis google group:
%   http://groups.google.com/group/ucsd-thesis
%
%
% ----------------------------------------------------------------------
% BUGS:
%
%   Please report all bugs at:
%   https://github.com/ucsd-thesis/ucsd-thesis/issues
%
%
% ----------------------------------------------------------------------
% More control of the formatting of your thesis can be achieved through
% modifications of the included LaTeX class files:
%
%   * ucsd.cls    -- Class file
%   * uct10.clo   -- Configuration files for font sizes 10pt, 11pt, 12pt
%     uct11.clo
%     uct12.clo
%
% ----------------------------------------------------------------------



% Setup the documentclass
% default options: 12pt, oneside, final
%
% fonts: 10pt, 11pt, 12pt -- are valid for UCSD dissertations.
% sides: oneside, twoside -- note that two-sided theses are not accepted
%                            by OGS.
% mode: draft, final      -- draft mode switches to single spacing,
%                            removes hyperlinks, and places a black box
%                            at every overfull hbox (check these before
%                            submission).
% chapterheads            -- Include this if you want your chapters to read:
%                              Chapter 1
%                              Title of Chapter
%
%                            instead of
%                              1 Title of Chapter
\documentclass[12pt,chapterheads]{ucsd}



% Include all packages you need here.
% Some standard options are suggested below.
%
% See the project wiki for information on how to use
% these packages. Other useful packages are also listed there.
%
%   http://code.google.com/p/ucsd-thesis/wiki/GettingStarted



%% AMS PACKAGES - Chances are you will want some or all
%    of these if writing a dissertation that includes equations.
%  \usepackage{amsmath, amscd, amssymb, amsthm}

%% BONUS MATH
%  \usepackage{mathtools}

%% MARGIN REQUIREMENTS IN TITLES - Hyphenation in a Section Title does not always respect margin settings in Latex.  To force no hyphentation, uncomment the package below.
%  \usepackage[raggedright]{titlesec}

%% GRAPHICX - This is the standard package for
%    including graphics for latex/pdflatex.
\usepackage{scrextend}
\usepackage{pslatex}
\usepackage{graphicx}
\usepackage{amsmath}
\usepackage{bm}

%% CAPTION
% This overrides some of the ugliness in ucsd.cls and
% allows the text to be double-spaced while letting figures,
% tables, and footnotes to be single-spaced--all OGS requirements.
% NOTE: Must appear after graphics and ams math
\makeatletter
\gdef\@ptsize{2}% 12pt documents
\let\@currsize\normalsize
\makeatother
\usepackage{setspace}
\doublespace
\usepackage[font=small, width=0.9\textwidth]{caption}

%% SUBFIG - Use this to place multiple images in a
%    single figure.  Subfig will handle placement and
%    proper captioning (e.g. Figure 1.2(a))
% \usepackage{subfig}

%% TIMES FONT - replacements for Computer Modern
%%   This package will replace the default font with a
%%   Times-Roman font with math support.
% \usepackage[T1]{fontenc}
% \usepackage{mathptmx}

%% INDEX
%   Uncomment the following two lines to create an index:
% \usepackage{makeidx}
% \makeindex
%   You will need to uncomment the \printindex line near the
%   bibliography to display the index.  Use the command
% \index{keyword}
%   within the text to create an entry in the index for keyword.
%   To compile a LaTeX document with an index the 'makeindex'
%   command will need to be run.  See the wiki for more details.

%% HYPERLINKS
%   To create a PDF with hyperlinks, you need to include the hyperref package.
%   THIS HAS TO BE THE LAST PACKAGE INCLUDED!
%   Note that the options plainpages=false and pdfpagelabels exist
%   to fix indexing associated with having both (ii) and (2) as pages.
%   Also, all links must be black according to OGS.
%   See: http://www.tex.ac.uk/cgi-bin/texfaq2html?label=hyperdupdest
%   Note: This may not work correctly with all DVI viewers (i.e. Yap breaks).
%   NOTE: hyperref will NOT work in draft mode, as noted above.
% \usepackage[colorlinks=true, pdfstartview=FitV,
%             linkcolor=black, citecolor=black,
%             urlcolor=black, plainpages=false,
%             pdfpagelabels]{hyperref}
% \hypersetup{ pdfauthor = {Your Name Here},
%              pdftitle = {The Title of The Dissertation},
%              pdfkeywords = {Keywords for Searching},
%              pdfcreator = {pdfLaTeX with hyperref package},
%              pdfproducer = {pdfLaTeX} }
% \urlstyle{same}
% \usepackage{bookmark}


%% CITATIONS
% Sets citation format
% and fixes up citations madness
\usepackage{microtype}  % avoids citations that hang into the margin


%% FOOTNOTE-MAGIC
% Enables footnotes in tables, re-referencing the same footnote multiple times.
\usepackage{footnote}
\makesavenoteenv{tabular}
\makesavenoteenv{table}


%% TABLE FORMATTING MADNESS
% Enable all sorts of fun table tricks
\usepackage{rotating}  % Enables the sideways environment (NCPW)
\usepackage{array}  % Enables "m" tabular environment http://ctan.org/pkg/array
\usepackage{booktabs}  % Enables \toprule  http://ctan.org/pkg/array

%%%%%%%%%%%%%%%%%%%%%%%%%%%%%%%%%%%%%%%%%%%%%%%%%%%%%%
% redefine the comma to allow breaks in inline math
\makeatletter
\def\old@comma{,}
\catcode`\,=13
\def,{%
  \ifmmode%
    \old@comma\discretionary{}{}{}%
  \else%
    \old@comma%
  \fi%
}
\makeatother
%%%%%%%%%%%%%%%%%%%%%%%%%%%%%%%%%%%%%%%%%%%%%%%%%%%%%%


\begin{document}

%% FRONT MATTER
%
%  All of the front matter.
%  This includes the title, degree, dedication, vita, abstract, etc..
%  Modify the file template_frontmatter.tex to change these pages.
% %
%
% UCSD Doctoral Dissertation Template
% -----------------------------------
% http://ucsd-thesis.googlecode.com
%
%


%% REQUIRED FIELDS -- Replace with the values appropriate to you

% No symbols, formulas, superscripts, or Greek letters are allowed
% in your title.
\title{Periodicity Transforms and their Applications to Digital Audio}

\author{Jacob Sundstrom}
\degreeyear{\the\year}

% Master's Degree theses will NOT be formatted properly with this file.
\degreetitle{Doctor of Philosophy}

\field{Music}
\specialization{Computer Music}  % If you have a specialization, add it here

\chair{Professor Miller Puckette}
% Uncomment the next line iff you have a Co-Chair
% \cochair{Professor Cochair Semimaster}
%
% Or, uncomment the next line iff you have two equal Co-Chairs.
%\cochairs{Professor Chair Masterish}{Professor Chair Masterish}

%  The rest of the committee members  must be alphabetized by last name.
\othermembers{
Professor Shahrokh Yadegari\\
Professor Anthony Burr\\
Professor Clinton Tolley\\
}
\numberofmembers{4} % |chair| + |cochair| + |othermembers|


%% START THE FRONTMATTER
%
\begin{frontmatter}

%% TITLE PAGES
%
%  This command generates the title, copyright, and signature pages.
%
\makefrontmatter

%% DEDICATION
%
%  You have three choices here:
%    1. Use the ``dedication'' environment.
%       Put in the text you want, and everything will be formated for
%       you. You'll get a perfectly respectable dedication page.
%
%
%    2. Use the ``mydedication'' environment.  If you don't like the
%       formatting of option 1, use this environment and format things
%       however you wish.
%
%    3. If you don't want a dedication, it's not required.
%
%
\begin{dedication}
    To two, the loneliest number since the number one.
\end{dedication}


% \begin{mydedication} % You are responsible for formatting here.
%   \vspace{1in}
%   \begin{flushleft}
% 	To me.
%   \end{flushleft}
%
%   \vspace{2in}
%   \begin{center}
% 	And you.
%   \end{center}
%
%   \vspace{2in}
%   \begin{flushright}
% 	Which equals us.
%   \end{flushright}
% \end{mydedication}



%% EPIGRAPH
%
%  The same choices that applied to the dedication apply here.
%
\begin{epigraph} % The style file will position the text for you.
  \emph{It is better to be roughly right than precisely wrong.}\\
  ---John Maynard Keynes
\end{epigraph}

% \begin{myepigraph} % You position the text yourself.
%   \vfil
%   \begin{center}
%     {\bf Think! It ain't illegal yet.}
%
% 	\emph{---George Clinton}
%   \end{center}
% \end{myepigraph}


%% SETUP THE TABLE OF CONTENTS
%
\tableofcontents
\listoffigures  % Comment if you don't have any figures
\listoftables   % Comment if you don't have any tables



%% ACKNOWLEDGEMENTS
%
%  While technically optional, you probably have someone to thank.
%  Also, a paragraph acknowledging all coauthors and publishers (if
%  you have any) is required in the acknowledgements page and as the
%  last paragraph of text at the end of each respective chapter. See
%  the OGS Formatting Manual for more information.
%
\begin{acknowledgements}
 Thanks to whoever deserves credit for Blacks Beach, Porters Pub, and
 every coffee shop in San Diego.

 Thanks also to hottubs.
\end{acknowledgements}


%% VITA
%
%  A brief vita is required in a doctoral thesis. See the OGS
%  Formatting Manual for more information.
%
\begin{vitapage}
\begin{vita}
  \item[2012] B.~A. in Music \emph{Honors with Distinction}, Minor in Philosophy, University of California, San Diego
  \item[2015] M.~M. in Music, University of Washington, Seattle
  \item[2022] Ph.~D. in Music, University of California, San Diego
\end{vita}
\begin{publications}
  \item Your Name, ``A Simple Proof Of The Riemann Hypothesis'', \emph{Annals of Math}, 314, 2007.
  \item Your Name, Euclid, ``There Are Lots Of Prime Numbers'', \emph{Journal of Primes}, 1, 300 B.C.
\end{publications}
\end{vitapage}


%% ABSTRACT
%
%  Doctoral dissertation abstracts should not exceed 350 words.
%   The abstract may continue to a second page if necessary.
%
\begin{abstract}
  This dissertation will be abstract.
\end{abstract}


\end{frontmatter}


%% DISSERTATION
% \chapter{Introduction}

In \cite{sethares1999periodicity}, W. A. Sethares and T. W. Staley introduced the concept of the ``periodicity transform" (PT) whereby a signal can be decomposed into periodic basis vectors by projection onto periodic subspaces. The PT excels in situations where the time series is described best in terms of period rather than frequency. This technique and its variations have been applied to astromonical data (\cite{buccheri1985time}), machine vibration (\cite{muresan2003orthogonal}), gene sequencing (\cite{arora2007detection}), and musical rhythms (\cite{sethares2001meter}); however, its applcation to time-domain audio signals directly has been lacking. Since the basis vectors calculated with the PT often have overlapping spectra, it seems a natural extention to apply the PT to both analysis and synthesis of audio, and especially to musical audio. In doing so, a short-time periodicity transform was developed in order to capture the changing periodic nature of musical signals and is applied to analyze and resynthesize an excerpt of a well-known electronic work.


Since Sethares and Staley in 1999, the PT has been developed in multiple directions. This chapter concerns itself with the examining and presenting existing versions and their applicability to audio analysis and synthesis.

\section{Nonorthogonal Periodic Decomposition}
Sethares and Staley

Original periodicity transform

Basis vectors are not orthogonal.

\section{Orthogonal Periodic Decomposition}

The orthogonal, exactly periodic decomposition of Muresan and Parks is based on the algorithm presented in \cite{wise1976maximum}.

In \cite{muresan2003orthogonal}, the process of orthogonal projection without constructing the orthogonal subspaces themselves is introduced. Let $x_p^\perp = \pi(x, P_p)^\perp$ be the orthogonal projection of $x$ onto subspace $P_p$.

To make $x_p \rightarrow x_p^\perp$, get the factors of $p$ so that $F_p = \{f : f|p, f \neq p\}$. Then project $x_p$ onto the subspaces corresponding with $F_p$, also subtracting out each of their factors as well from their own projections:
\begin{align}\label{intro:orthgonalProjection}
    \pi(x, P_p)^\perp = x_p - \sum_{f \in F_p} \pi(x_p, P_f)^\perp
\end{align}
where $x_p = \pi(x, P_p)$.

It turns out, however, that since $P_f \in P_p$ for $f|p$, it is not necessary to single out each factor of $p$ before subtracting out each projection. One can simply subtract the non-orthogonalized projection $\pi(x_p, P_f)$ from $x_p$ for $f \in F_p$ to get $x_p^\perp$. Depending on the order, however, one will get projections equal to 0 for periods $d$ if $d|f$ and $f$ has already been subtracted. For instance:
\begin{align*}
    x_{25} &= \pi(x, P_{25}) \\
    \pi(x_{25}, P_5) &= 0
\end{align*}

In order to avoid projecting and subtracting vectors that will equal zero, it is possible to project only at periods $p/f'$, where $f'$ is a prime factor. We then redefine Equation \eqref{intro:orthgonalProjection}:
\begin{align}
    \pi(x, P_p)^\perp = x_p - \sum_{f \in F'_p} \pi \bigg( x_p, P_{\frac{p}{f}} \bigg)
\end{align}
where $F'_p = F_p \cap \mathbb{P}$ and $\mathbb{P}$ is the set of all prime numbers. A proof of this is given in Appendix XXX.


\section{Ramanujan Periodic Decomposition}

Vaidyanathan and Tenneti

Struggles hardcore with non-integer periods. Suggestion when doing machine vibration is to adjust the samplerate so that the period aligns to an factor of the samplerate. This begs the question why one would have the need to search for a period one already knows is there. That is, without knowing the period a priori it would be impossible to adjust the samplerate accordingly.

\chapter{Quadratically Optimized Periodic Decomposition} \label{chap:qo}

The fundamental difference between this and \cite{sethares1999periodicity} is the difference in computing the residual. That is, one can frame the residual computation as a quadratic program which allows one to more accurately and find the periodic components.

\section{Find the matrix $\bm{A}$}
Assume we have the following:
\begin{align*}
x &= \text{input sequence} \\
Q &= \text{set of periods posited to be in } x \\
f(n) &= \{f \in N : f|n \} \text{ (i.e. a function that the set of all factors of $n$)} \\
\varphi(n) &= \text{Euler's totient function}
\end{align*}

Let $\bm{A}$ be a ``basis matrix'', or a matrix of linearly independent columns which span the space of the sequence $x$ with periods $Q$. The column dimensionality, $c_{\bm{A}}$, of $\bm{A}$ is:
\begin{align*}
    c_{\bm{A}} = \sum_{R_i \in R} \varphi(R_i)
\end{align*}
where $R$ is the union of all the factors of the periods in $Q$:
\begin{align*}
    R = \bigcup_{q_i \in Q} f(q_i)
\end{align*}

In creating the basis matrix $\bm{A}$, one must ensure that each periodic component is counted \emph{once}. That is, each periodic component $q$ also contains the components $d|q$ and one must construct the basis matrix with this in mind, taking care to remove the component from a subspace if it is already present. The method for doing this is as follows:




Define $P$:
\begin{align*}
    P &= \Bigg( \bigcup_{\forall p, q \in Q} f(p) \cap f(q) \Bigg) \cup Q
\end{align*}
i.e. the union of the intersections of the factors of all the elements in $Q$.

To find the column dimensionality, $c_{p}$, of the basis matrix $\bm{B_p}$ for $p \in P$:
\begin{align*}
    % c_{p} &= p - \sum_{f : f(p) \cap P \setminus p} \varphi(f)
    c_{p} &= p - \sum_{d|p \in P \setminus p} \varphi(d)
\end{align*}

%% If N is the columns
% We then define $\bm{A}$ as:
% \begin{align*}
%     \bm{A} &= \begin{bmatrix}
%             \bm{B}_{p_0} \\
%             \bm{B}_{p_1} \\
%             \vdots \\
%             \bm{B}_{p_n} \\
%         \end{bmatrix}
%         , \text{ } \forall p \in P
% \end{align*}
% where there are $n$ elements in $\bm{P}$.

We then define $\bm{A}$ as:
\begin{align*}
    \bm{A} &= \begin{bmatrix}
            \bm{B}_{p_0} &
            \bm{B}_{p_1} &
            \hdots &
            \bm{B}_{p_n}
        \end{bmatrix}
        , \text{ } \forall p \in P
\end{align*}
where there are $n$ elements in $\bm{P}$.

To find the total column dimensionality, $c_{\boldsymbol{A}}$ of the matrix $\bm{A}$:
\begin{align*}
    c_{\bm{A}} &= \sum_{p \in P} c_p
\end{align*}

% In this way, we can be assured that not only are the basis vectors in $\bm{A}$ linearly independent, but also that all traces of some period $q$ are removed from the residual signal $h$.

\subsection{This also works}
We can also get a basis matrix for the periods in $q$ by using the same equation as above that uses $varphi(n)$ by creating $\bm{A}$ along the way and keeping track of which periods (and factors of the periods) we've added along the way.

Let $F = \emptyset$. To find the column dimensionality, $c_q$, for $q \in Q$:
\begin{align}
    c_q &= q - \sum_{d|q \in F} \varphi(d)
\end{align}

We then add elements to $F$ every time we add a basis matrix $\bm{B}_q$ to $\bm{A}$:
\begin{align*}
    F &= \{ F \cup f(q) \}_{\neq}
\end{align*}
i.e. every time we process a period $q$, add all the factors of $q$ to $F$, removing duplicates.

Repeat this process for every $q \in Q$, adding to $F$ along the way. (This means that we need to keep track of which factors we've added along the way which makes it uglier in some ways, better in others.)

Therefore, we then define $\bm{A}$ as:
\begin{align*}
    \bm{A} &= \begin{bmatrix}
            \bm{B}_{q_0} &
            \bm{B}_{q_1} &
            \hdots &
            \bm{B}_{q_n}
        \end{bmatrix}
        , \text{ } \forall q \in \bm{Q}
\end{align*}
where there are $n$ elements in $\bm{Q}$.

\section{Not a real section but describing getting the real periods back}

\chapter{Decomposing and Resynthesizing Audio}
An elementary use of the PT in audio is for the decomposition and resynthesis of audio signals.

In contrast to other literature where it is sufficient to detect the presence of periodicities in a signal, the manipulation of audio requires one to retrieve the periodic waveforms themselves.

A requirement of a PT on audio data is its ability to handle non-integer periods, in contrast to something like DNA sequencing where periods are guarenteed to be integers.

\section{Synthetic Audio}

\subsection{Sine Waves}

\subsection{Sine Waves + Noise}

\subsection{Mixed Waves}
An interesting phenomenon that is not necessarily unexpected is ``waveform interference'' whereby discovered waveforms bear features of other waveforms in the signal. Below is a signal comprised of a sine ($p = 100$), a square ($p = 171$), and a sawtooth ($p = 203$) which was analyzed using the QO method:

Figure XXX shows the orignal periods compared against the periodic components derived from the QO. Note that despite their mututal prime-ness, the waveforms ``interfere'' with one another with regard to their shapes.




\section{Recorded Audio}


\section{PEAQ Comparisons}

% \chapter{Decomposing and Resynthesizing Audio}


\subsection{Synthetic Audio}

\subsection{Recorded Audio}

% \chapter{Applications Beyond Analysis and Resynthesis}



\subsection{Extraction of the Periodic Waveforms}
The preceeding chapters almost exclusively examined the \emph{detection} of periodic sequences in a signal but in only a few instances are the periodic subsequences themselves revealed for further processing. Namely, while the method of Ramanujan sequences is extremely powerful in revealing the existence of periodic subsequences, the sequences themselves are difficult to recover.
Likewise, the orthogonal, exactly periodic decomposition of \cite{muresan2003orthogonal} is less useful in audio-space since it is often the case that one desires the harmonic components to be included in the extraction of the waveforms. In order to recover the waveform, additional steps must be undertaken and the result is questionable.

In the method presented in \cite{sethares1999periodicity}, the waveform is derived \emph{first} and is therefore immediately available. In the quadratic optimization method presented in Chapter \ref{chap:qo}, it is relatively straightforward to derive the periodic subsequences by means of a similar process.

\subsection{Time Stretching}

\subsection{Monaural Source Separation}


%% END MATTER


% \printindex %% Uncomment to display the index
% \nocite{}  %% Put any references that you want to include in the bib
%               but haven't cited in the braces.
\bibliographystyle{alpha}  %% This is just my personal favorite style.
%                              There are many others.
%\setlength{\bibleftmargin}{0.25in}  % indent each item
%\setlength{\bibindent}{-\bibleftmargin}  % unindent the first line
%\def\baselinestretch{1.0}  % force single spacing
%\setlength{\bibitemsep}{0.16in}  % add extra space between items
\bibliography{template}  %% This looks for the bibliography in template.bib
%                          which should be formatted as a bibtex file.
%                          and needs to be separately compiled into a bbl file.
\singlespace  %to force bibilography environment to use single spacing for each entry
              %double spacing between entries remains
\end{document}
